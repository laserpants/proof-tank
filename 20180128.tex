\documentclass{article}

\usepackage[margin=0.7in]{geometry}
\usepackage[utf8]{inputenc}
  
\usepackage{amsmath,amssymb,amsfonts,amsthm}

\newenvironment{itemized}{ 
\begin{itemize}
\setlength{\itemsep}{0pt}
\setlength{\parskip}{0pt}
\setlength{\parsep}{0pt}     
}{\end{itemize}} 

\begin{document}

\section*{K-Algebras}

\subsection*{Vector spaces}

A vector space $V$ over a field $\mathbb{K}$ is a set of vectors, together with two operations; 

  \begin{itemized}
    \item vector addition; $ (+) : V \times V \rightarrow V $; and
    \item scalar multiplication; $ (\cdot) : \mathbb{K} \times V \rightarrow V $;
  \end{itemized}

such that 
  
  \begin{itemized}
    \item The set $V$ forms an abelian group under vector addition;
      \begin{itemized}
        \item Associativity; $ \forall u, v, w \in V : (u + v) + w = u + (v + w) $
        \item Commutativity; $ \forall u, v \in V : u + v = v + u $
        \item Identity; $ \exists 0 \in V $ s.t. $ \forall v \in V : v + 0 = v $
        \item Inverse; $ \forall v \in V, \exists (-v) \in V $ s.t. $ v + (-v) = 0 $
      \end{itemized}
    \item Scalar multiplication distributes over vector addition; $ \forall a \in \mathbb{K}, \vec{u}, \vec{v} \in V : a \cdot (\vec{u} + \vec{v}) = (a \cdot \vec{u}) + (a \cdot \vec{v}) $
    \item Scalar multiplication distributes over addition in $\mathbb{K}$; $ \forall a, b \in \mathbb{K}, \vec{v} \in V : (a + b) \cdot \vec{v} = (a \cdot \vec{v}) + (b \cdot \vec{v}) $
    \item Scalar multiplication is \begin{em}compatible\end{em} with multiplication in $\mathbb{K}$; $ \forall a,b \in \mathbb{K}, \vec{v} \in V : a \cdot (b \cdot \vec{v}) = (ab) \cdot \vec{v} $ 
    \item The identity law for scalar multiplication holds: $ \forall v \in V : 1_{\mathbb{K}} \cdot v = v $
  \end{itemized}

\subsubsection*{Linear map}

If $V$ and $W$ are vector spaces defined over $\mathbb{K}$, a function $f : V \rightarrow W$ is a \begin{em}linear map\end{em} if the following holds

  \begin{itemized}
    \item $ \forall \vec{u}, \vec{v} \in V : f(\vec{u} + \vec{v}) = f(\vec{u}) + f(\vec{v}) $
    \item $ \forall a \in \mathbb{K}, v \in V : f(a \cdot \vec{v}) = a \cdot f(\vec{v}) $
  \end{itemized}

\subsubsection*{Bilinear map}

Given three vector spaces, $U, V, $ and $W$, defined over a a field $\mathbb{K}$,

\subsection*{Algebra over a field}

If $\mathbb{K}$ is a field, and $V$ a vector space over $\mathbb{K}$ equipped with a bilinear map $(\cdot) : V \times V \rightarrow V$, then $V$ is called an \begin{em}algebra over\end{em} $\mathbb{K}$ (or $K$-algebra for short).

  TODO

\end{document}
