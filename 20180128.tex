\documentclass{article}

\usepackage[margin=0.7in]{geometry}
\usepackage[utf8]{inputenc}
  
\usepackage{amsmath,amssymb,amsfonts,amsthm}

\newenvironment{itemized}{ 
\begin{itemize}
\setlength{\itemsep}{0pt}
\setlength{\parskip}{0pt}
\setlength{\parsep}{0pt}     
}{\end{itemize}} 

\newenvironment{enumerated}{ 
\begin{enumerate}
\setlength{\itemsep}{0pt}
\setlength{\parskip}{0pt}
\setlength{\parsep}{0pt}     
}{\end{enumerate}} 

\renewcommand{\vec}[1]{\mathbf{#1}}

\begin{document}

\section*{K-Algebra}

\subsection*{Vector space}

A vector space $V$ over a field $\mathbb{K}$ is a set of vectors, together with two operations; 

  \begin{itemized}
    \item vector addition; $ (+) : V \times V \rightarrow V $; and
    \item scalar multiplication; $ (\cdot) : \mathbb{K} \times V \rightarrow V $
  \end{itemized}

such that 
  
  \begin{itemized}
    \item The set $V$ forms an abelian group under the operation of vector addition;
      \begin{itemized}
        \item Associativity; $ \forall \vec{u}, \vec{v}, \vec{w} \in V : (\vec{u} + \vec{v}) + \vec{w} = \vec{u} + (\vec{v} + \vec{w}) $
        \item Commutativity; $ \forall \vec{u}, \vec{v} \in V : \vec{u} + \vec{v} = \vec{v} + \vec{u} $
        \item Identity; $ \exists \vec{0} \in V $ s.t. $ \forall \vec{v} \in V : \vec{v} + \vec{0} = \vec{v} $
        \item Inverse; $ \forall \vec{v} \in V, \exists \vec{(-v)} \in V $ s.t. $ \vec{v} + \vec{(-v)} = \vec{0} $
      \end{itemized}
    \item Scalar multiplication distributes over vector addition; $ \forall a \in \mathbb{K}, \vec{u}, \vec{v} \in V : a \cdot (\vec{u} + \vec{v}) = (a \cdot \vec{u}) + (a \cdot \vec{v}) $
    \item Scalar multiplication distributes over addition in $\mathbb{K}$; $ \forall a, b \in \mathbb{K}, \vec{v} \in V : (a + b) \cdot \vec{v} = (a \cdot \vec{v}) + (b \cdot \vec{v}) $
    \item Scalar multiplication is \begin{em}compatible\end{em} with multiplication in $\mathbb{K}$; $ \forall a,b \in \mathbb{K}, \vec{v} \in V : a \cdot (b \cdot \vec{v}) = (ab) \cdot \vec{v} $ 
    \item The following identity law holds for scalar multiplication; $ \forall \vec{v} \in V : 1_{\mathbb{K}} \cdot \vec{v} = \vec{v} $ (where $1_{\mathbb{K}}$ is the multiplicative identity in $\mathbb{K}$)
  \end{itemized}

\subsubsection*{Compact definition}

TODO

\subsubsection*{Generalization to rings}

The notion of a \begin{em}module\end{em} is a generalization of vector spaces, in which the scalars are elements of any ring (i.e., not necessarily a field).

\subsection*{Linear map}

If $V$ and $W$ are vector spaces defined over $\mathbb{K}$, a function $f : V \rightarrow W$ is a \begin{em}linear map\end{em} if it preserves the vector space structure under the two operations, i.e,

  \begin{enumerated}
    \item vector addition; $ \forall \vec{u}, \vec{v} \in V : f(\vec{u} + \vec{v}) = f(\vec{u}) + f(\vec{v}) $; and
    \item scalar multiplication; $ \forall a \in \mathbb{K}, \vec{v} \in V : f(a \cdot \vec{v}) = a \cdot f(\vec{v}) $
  \end{enumerated}

In other words, a linear map is a \begin{em}homomorphism\end{em} between vector spaces.

\subsection*{Bilinear map}

Given three vector spaces, $U, V, $ and $W,$ all defined over a a field $\mathbb{K}$, a \begin{em}bilinear map\end{em} is a function $b : U \times V \rightarrow W$ which is linear in both arguments. That is, $b$ is linear with respect to;

  \begin{itemized}
    \item addition, in the first argument; $ \forall \vec{u}, \vec{x} \in U, \vec{v} \in V : b(\vec{u} + \vec{x}, \vec{v}) = b(\vec{u}, \vec{v}) + b(\vec{x}, \vec{v}) $
    \item addition, in the second argument; $ \forall \vec{u} \in U, \vec{v}, \vec{y} \in V : b(\vec{u}, \vec{v} + \vec{y}) = b(\vec{u}, \vec{v}) + b(\vec{u}, \vec{y}) $
    \item scalar multiplication; $ \forall c \in \mathbb{K}, \vec{u} \in U, \vec{v} \in V : b(c \cdot \vec{u}, \vec{v}) = c \cdot b(\vec{u}, \vec{v}) = b(\vec{u}, c \cdot \vec{v}) $
  \end{itemized}

\subsection*{Algebra over a field}

If $\mathbb{K}$ is a field, and $V$ a vector space over $\mathbb{K}$ equipped with a bilinear map $(\cdot) : V \times V \rightarrow V$, then $V$ is called an \begin{em}algebra over\end{em} $\mathbb{K}$ (or $K$-algebra for short).

  TODO

\end{document}
