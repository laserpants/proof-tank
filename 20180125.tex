\documentclass{article}

\usepackage[margin=0.7in]{geometry}
\usepackage[utf8]{inputenc}
  
\usepackage{amsmath,amssymb,amsfonts,amsthm}

\theoremstyle{definition}
\newtheorem{theorem}{Theorem}

\theoremstyle{definition}
\newtheorem{definition}{Definition}

\newtheorem{corollary}{Corollary}[theorem]
\newtheorem{lemma}[theorem]{Lemma}

\begin{document}

\begin{definition}
Given two integers $a$ and $b$, a number $s$ is the \begin{em}greatest common divisor\end{em} of $a$ and $b$ if (1) $s$ divides both $a$ and $b$, and (2) given any integer $t$, if $t$ is a common divisor of $a$ and $b$, then $t$ also divides $s$. In symbols;
\begin{enumerate}
  \item $s \mid a \wedge s \mid b$
  \item $\forall t \in \mathbb{Z} : (t \mid a \wedge t \mid b) \implies t \mid s$
\end{enumerate}
\end{definition}

\begin{theorem}
Every pair of non-zero integers has a \begin{em}greatest common divisor\end{em}. Furthermore, if $t = \gcd (a, b)$, then this number $t$ can be expressed as a \begin{em}linear combination\end{em} of $a$ and $b$. That is,
$$t = xa + yb \text{ \quad for some $x, y \in \mathbb{Z}$. }$$
\end{theorem}

\begin{proof}
Let $J$ be the set of all linear combinations of $a$ and $b$.
$$J = \{ xa + yb \mid x, y \in \mathbb{Z} \}$$
Since $a, b, x, y \in \mathbb{Z}$, it follows that $J$ is a subset of $\mathbb{Z}$. In fact, $J$ is an \begin{em}ideal\end{em} of $\mathbb{Z}$. To show this, we need to verify three conditions;

\begin{em}Condition 1.\end{em} $J$ is closed under addition:
$$(x_1a + y_1b) + (x_2a + y_2b) = (x_1 + x_2)a + (y_1 + y_2)b$$
which is in $J$ since both $(x_1 + x_2)$ and $(y_1 + y_2)$ are in $\mathbb{Z}$.

\begin{em}Condition 2.\end{em} $J$ is closed under inverses:
$$-(xa + yb) = (-x)a + (-y)b$$
which is in $J$ as well, since $-x, -y \in \mathbb{Z}$.

\begin{em}Condition 3.\end{em} $J$ absorbs products:
$$z(xa + yb) = (zx)a + (zy)b$$
where $z$ is an integer, and therefore so are $zx$ and $zy$.

Every ideal of $\mathbb{Z}$ is principal. If $J$ is principal, this means that $ J = \langle p \rangle $ for some $p = qa + rb$, where $q, r \in \mathbb{Z}$. Therefore $p$ divides any element of $J$, and $a$ and $b$ are in $J$, so
$$p \mid a \quad \text{ and } \quad p \mid b.$$

In other words, $p$ is a common divisor of $a$ and $b$. Next, to show that $p$ is also the \begin{em}greatest\end{em} common divisor; if $u$ is a common divisor of $a$ and $b$, then $a = ku$ and $b = lu$ for some integers $k$, and $l$. Then
$$p = qa + rb = qku + rlu = u(qk + rl)$$
which means that $u$ divides $p$. So, $p$ is indeed the \begin{em}gcd\end{em} of $a$ and $b$.
\end{proof}

\end{document}
