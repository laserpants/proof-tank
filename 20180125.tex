\documentclass{article}

\usepackage[margin=0.7in]{geometry}
\usepackage[utf8]{inputenc}
  
\usepackage{amsmath,amssymb,amsfonts,amsthm}

\theoremstyle{definition}
\newtheorem{theorem}{Theorem}

\theoremstyle{definition}
\newtheorem{definition}{Definition}

\newtheorem{corollary}{Corollary}[theorem]
\newtheorem{lemma}{Lemma}

\newenvironment{enumerated}{ 
\begin{enumerate}
\setlength{\itemsep}{0pt}
\setlength{\parskip}{0pt}
\setlength{\parsep}{0pt}     
}{\end{enumerate}} 

\begin{document}

\begin{definition}[Ideal]
  Given a commutative ring $R$, an \begin{em}ideal\end{em} of $R$ is a subset $I \subseteq R$ which satisfies the following conditions;
  \begin{enumerated}
    \item $I$ is closed under addition; $\forall x, y \in I : x + y \in I$
    \item $I$ is closed with respect to inverses; $\forall x \in I : (-x) \in I$
    \item $I$ absorbs products; $\forall x \in I, z \in R : xz \in I$
  \end{enumerated}
\end{definition}

\begin{definition}[Principal ideal] 
  An ideal $P$ of a ring $R$ is called a \begin{em}principal ideal\end{em} when there is some element $a$ in $R$, such that 
  $$P = aR = \{ar : r \in R\}.$$
  We say that the ideal $P$ is \begin{em}generated by\end{em} the element $a$, and use the notation $P = \langle a \rangle$.
\end{definition}

\begin{lemma}
\label{IdealsOfIntegers}
  Every ideal of $\mathbb{Z}$ is principal. (Another way to express this is to say that the integers form a \begin{em}principal ideal domain\end{em}, or PID.)
\end{lemma}
\begin{proof}
  Let $I \subseteq \mathbb{Z}$ be an ideal. If $I = \{0\}$, then $I$ is the principal ideal generated by $0$. If $I \ne \{0\}$, then let $m$ be the least positive element of $I$. We will find that $I = \langle m \rangle$. First, we know that $\langle m \rangle \subseteq I$, since $\langle m \rangle = \{mz : z \in \mathbb{Z}\}$ and $xz \in I$ for all $x \in I, z \in \mathbb{Z}$ (by the absorption property). Next, given an arbitrary element $n \in I$, applying Euclidean division we can write
  $$n = mq + r$$
  where $q, r \in \mathbb{Z}$ and $0 \le r < m$. So, $r = n - mq \in I$. It then immediately follows that $r = 0$, since $r < m$, and $m$ is the least positive element of $I$. Therefore, $n = mq + 0 = mq \in \langle m \rangle$, and since $n$ was chosen arbitrarily; if an element is in $I$, then it is in $\langle m \rangle$. This is the same as saying that $I \subseteq \langle m \rangle$, and we conclude that $I = \langle m \rangle$.
\end{proof}

\begin{definition}
  Given two integers $a$ and $b$, a number $s$ is the \begin{em}greatest common divisor\end{em} of $a$ and $b$ if (1) $s$ divides both $a$ and $b$, and (2) given any integer $t$, if $t$ is a common divisor of $a$ and $b$, then $t$ also divides $s$. In symbols;
  \begin{enumerated}
    \item $s \mid a \wedge s \mid b$
    \item $\forall t \in \mathbb{Z} : (t \mid a \wedge t \mid b) \implies t \mid s$
  \end{enumerated}
\end{definition}

\begin{theorem} 
  Every pair of non-zero integers has a \begin{em}greatest common divisor\end{em}. Furthermore, if $t = \gcd (a, b)$, then this number $t$ can be expressed as a \begin{em}linear combination\end{em} of $a$ and $b$. That is, 
  $$t = xa + yb \text{ \quad for some $x, y \in \mathbb{Z}$. }$$
\end{theorem}

\begin{proof} 
  Let $J$ be the set of all linear combinations of $a$ and $b$. 
  $$J = \{ xa + yb : x, y \in \mathbb{Z} \}$$ 
  Since $a, b, x, y \in \mathbb{Z}$, it follows that $J$ is a subset of $\mathbb{Z}$. In fact, $J$ is an \begin{em}ideal\end{em} of $\mathbb{Z}$. To show this, we verify the three conditions;

  \begin{enumerate}
    \item $J$ is closed under addition: 
      $$(x_1a + y_1b) + (x_2a + y_2b) = (x_1 + x_2)a + (y_1 + y_2)b$$
      which is in $J$ since both $(x_1 + x_2)$ and $(y_1 + y_2)$ are in $\mathbb{Z}$.
    \item $J$ is closed under inverses: 
      $$-(xa + yb) = (-x)a + (-y)b$$ 
      which is in $J$ as well, since $-x, -y \in \mathbb{Z}$.
    \item $J$ absorbs products: 
      $$z(xa + yb) = (zx)a + (zy)b$$ 
      where $z$ is an integer, and therefore so are $zx$ and $zy$.
  \end{enumerate}

  Lemma~\ref{IdealsOfIntegers} establishes that every ideal of $\mathbb{Z}$ is principal. If $J$ is principal, this means that $ J = \langle p \rangle $ for some $p = qa + rb$, where $q, r \in \mathbb{Z}$. Since $p$ divides every element of $J$, and $a = 1a + 0b$ and $b = 0a + 1b$ are in $J$, 
  $$p \mid a \quad \text{ and } \quad p \mid b.$$

  In other words, $p$ is a common divisor of $a$ and $b$. Next, to show that $p$ is the \begin{em}greatest\end{em} common divisor; if $u$ is a common divisor of $a$ and $b$, then $a = ku$ and $b = lu$ for some integers $k$, and $l$. Then
  $$p = qa + rb = qku + rlu = u(qk + rl)$$
  which means that $u$ divides $p$. So, $p$ is indeed the \begin{em}gcd\end{em} of $a$ and $b$.
\end{proof}

\end{document}
