\documentclass{article}

\usepackage[margin=0.7in]{geometry}
\usepackage[utf8]{inputenc}

\usepackage{amsmath,amssymb,amsfonts,amsthm}

\DeclareMathOperator{\End}{End}
\DeclareMathOperator{\Hom}{Hom}
\DeclareMathOperator{\id}{id}

\theoremstyle{definition}
\newtheorem{theorem}{Theorem}

\theoremstyle{definition}
\newtheorem{definition}{Definition}

\newtheorem{corollary}{Corollary}[theorem]
\newtheorem{lemma}{Lemma}

\newenvironment{itemized}{
\begin{itemize}
\setlength{\itemsep}{0pt}
\setlength{\parskip}{0pt}
\setlength{\parsep}{0pt}
}{\end{itemize}}

\newenvironment{enumerated}{
\begin{enumerate}
\setlength{\itemsep}{0pt}
\setlength{\parskip}{0pt}
\setlength{\parsep}{0pt}
}{\end{enumerate}}

\renewcommand{\vec}[1]{\mathbf{#1}}
\newcommand{\iu}{{i\mkern1mu}}

\begin{document}

\begin{lemma} 
  If $F$ is a field, then $F[x]$ forms an integral domain.
\end{lemma} 
\begin{proof}
  First of all, if $F$ is a field, then $F[x]$ is a commutative ring with unity.
  \begin{itemize}
    \item $F[x]$ is closed under addition:
      \begin{center}
        \begin{tabular}{r@{ } l}
          $ \forall a(x), b(x) \in F[x] : $ & $ a(x) + b(x) $ \\
          = & $ (a_0 + a_1x + a_2x^2 + \dots + a_nx^n) + (b_0 + b_1x + b_2x^2 + \dots + b_nx^n) $ \\
          = & $ (a_0 + b_0) + (a_1 + b_1)x + (a_2 + b_2)x^2 + \dots + (a_n + b_n)x^n $ \\
          $ \in $ & $ F[x] $ (where $n = \max[\deg a(x), \deg b(x)])$.
        \end{tabular}
      \end{center}
    \item Addition is associative:
      \begin{tabular}{l}
        $ [a(x) + b(x)] + c(x) $
      \end{tabular}
    \item Existence of additive identity:
      \begin{tabular}{l@{ } l}
        $ \forall a(x) \in F[x] : a(x) + 0 $ & $ = (a_0 + a_1x + a_2x^2 + \cdots + a_nx^n) + 0 $ \\
        & $ = (a_0 + 0) + a_1x + a_2x^2 + \cdots + a_nx^n $ \\
        & $ = a(x) $
      \end{tabular}
    \item Existence of additive inverses:
    \item Addition is commutative:
      \begin{tabular}{l@{ } l}
        $ \forall a(x), b(x) $ & $ \in F[x] : $ \\
        $ a(x) + b(x) $ & $ = (a_0 + a_1x + a_2x^2 + \dots + a_nx^n) + (b_0 + b_1x + b_2x^2 + \dots + b_nx^n) $ \\
        & $ = (a_0 + b_0) + (a_1 + b_1)x + (a_2 + b_2)x^2 + \dots + (a_n + b_n)x^n $ \\ 
        & $ = (b_0 + a_0) + (b_1 + a_1)x + (b_2 + a_2)x^2 + \dots + (b_n + a_n)x^n $ \\ 
        & $ = (b_0 + b_1x + b_2x^2 + \dots + b_nx^n) + (a_0 + a_1x + a_2x^2 + \dots + a_nx^n) $ \\
        & $ = b(x) + a(x) $
      \end{tabular}
  \end{itemize}
\end{proof}

\begin{lemma} 
  $F[x]$ is Euclidean.
% https://drexel28.wordpress.com/2011/10/24/polynomial-rings-in-relation-to-euclidean-domains-pids-and-ufds/
\end{lemma} 

\begin{theorem} 
  If $a(x)$ is a polynomial over some field $F$, then $x - c$ is a factor of $a(x)$ precisely when $c$ is a root of $a(x)$.
\end{theorem}
If $x - c$ is a factor of $a(x)$, this means that there is some polynomial $b(x)$, such that $a(x) = (x - c)b(x)$. Then $a(c) = (c - c)b(c) = 0 b(c) = 0$. This shows that $c$ is a root of $a(x)$.

Assume that $c$ is a root of $a(x)$. Then we use Euclidean division and write 
$$ a(x) = q(x)(x - c) + r(x) $$


% If $a(x)$ is a polynomial, and $c$ is a root of $a(x)$, then $x - c$ is a factor of $a(x)$.
% 
% $$
% a(x) = b(x)(x - c) + r(x)
% $$
% 
% $$a(c) = 0$$
% 
% $$ a(x) = (x - c)b(x) $$
% 
% 
% If $a(x)$ is a polynomial, and $c_1, c_2, \dots, c_n$ are distinct roots of
% $a(x)$, then $(x - c_1)(x - c_2)\dots(x - c_n)$ is a factor of $a(x)$.

\end{document}
