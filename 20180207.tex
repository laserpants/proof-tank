\documentclass{article}

\usepackage[margin=0.7in]{geometry}
\usepackage[utf8]{inputenc}

\usepackage{amsmath,amssymb,amsfonts,amsthm}

\DeclareMathOperator{\End}{End}
\DeclareMathOperator{\Hom}{Hom}
\DeclareMathOperator{\id}{id}

\theoremstyle{definition}
\newtheorem{theorem}{Theorem}

\theoremstyle{definition}
\newtheorem{definition}{Definition}

\newtheorem{corollary}{Corollary}[theorem]
\newtheorem{lemma}{Lemma}

\newenvironment{itemized}{
\begin{itemize}
\setlength{\itemsep}{0pt}
\setlength{\parskip}{0pt}
\setlength{\parsep}{0pt}
}{\end{itemize}}

\newenvironment{enumerated}{
\begin{enumerate}
\setlength{\itemsep}{0pt}
\setlength{\parskip}{0pt}
\setlength{\parsep}{0pt}
}{\end{enumerate}}

\renewcommand{\vec}[1]{\mathbf{#1}}
\newcommand{\iu}{{i\mkern1mu}}

\begin{document}

\begin{lemma} 
  If $F$ is a field, then $F[x]$ forms an integral domain.
\end{lemma} 
\begin{proof}
  First of all, if $F$ is a field, then $F[x]$ is a commutative ring with unity.
\end{proof}

\begin{theorem} 
  If $a(x)$ is a polynomial over some field $F$, then $x - c$ is a factor of $a(x)$ precisely when $c$ is a root of $a(x)$.
\end{theorem}
If $x - c$ is a factor of $a(x)$, this means that there is some polynomial $b(x)$, such that $a(x) = (x - c)b(x)$. Then $a(c) = (c - c)b(c) = 0 b(c) = 0$. This shows that $c$ is a root of $a(x)$.

If $c$ is a root of $a(x)$, we can use euclidean division and write 
$$ a(x) = q(x)(x - c) + r(x) $$


% If $a(x)$ is a polynomial, and $c$ is a root of $a(x)$, then $x - c$ is a factor of $a(x)$.
% 
% $$
% a(x) = b(x)(x - c) + r(x)
% $$
% 
% $$a(c) = 0$$
% 
% $$ a(x) = (x - c)b(x) $$
% 
% 
% If $a(x)$ is a polynomial, and $c_1, c_2, \dots, c_n$ are distinct roots of
% $a(x)$, then $(x - c_1)(x - c_2)\dots(x - c_n)$ is a factor of $a(x)$.

\end{document}
