\documentclass{article}

\usepackage[margin=0.7in]{geometry}
\usepackage[utf8]{inputenc}

\usepackage{amsmath,amssymb,amsfonts,amsthm}

\DeclareMathOperator{\End}{End}
\DeclareMathOperator{\Hom}{Hom}
\DeclareMathOperator{\id}{id}

\newenvironment{itemized}{
\begin{itemize}
\setlength{\itemsep}{0pt}
\setlength{\parskip}{0pt}
\setlength{\parsep}{0pt}
}{\end{itemize}}

\newenvironment{enumerated}{
\begin{enumerate}
\setlength{\itemsep}{0pt}
\setlength{\parskip}{0pt}
\setlength{\parsep}{0pt}
}{\end{enumerate}}

\renewcommand{\vec}[1]{\mathbf{#1}}
\newcommand{\iu}{{i\mkern1mu}}

\begin{document}

\subsection*{Vector spaces}
A \begin{em}vector space\end{em} defined over a field $\mathbb{K}$ is a set $V$ of elements, known as \begin{em}vectors\end{em}, together with two operations;
\begin{itemized}
  \item vector addition, $+ : V \times V \rightarrow V$; and
  \item scalar multiplication, $\cdot : \mathbb{K} \times V \rightarrow V;$
\end{itemized}
such that the following eight axioms are satisfied:
\begin{enumerated}
  \item Addition of vectors is associative:
  $$ \forall \vec{u}, \vec{v}, \vec{w} \in V : (\vec{u} + \vec{v}) + \vec{w} = \vec{u} + (\vec{v} + \vec{w}) $$
  \item Addition of vectors is commutative:
  $$ \forall \vec{u}, \vec{v} \in V : \vec{u} + \vec{v} = \vec{v} + \vec{u} $$
  \item There exists an identity vector in $V$:
  $$ \exists \vec{0} \in V \text{ s.t. } \forall \vec{v} \in V : \vec{v} + \vec{0} = \vec{v} $$
  \item Every vector has an additive inverse:
  $$ \forall \vec{v} \in V, \exists \vec{(-v)} \in V \text{ s.t. } \vec{v} + \vec{(-v)} = \vec{0} $$
  \item Scalar multiplication distributes over vector addition:
  $$ \forall a \in \mathbb{K}, \vec{u}, \vec{v} \in V : a \cdot (\vec{u} + \vec{v}) = (a \cdot \vec{u}) + (a \cdot \vec{v}) $$
  \item Scalar multiplication is \begin{em}compatible\end{em} with multiplication in $\mathbb{K}$:
  $$ \forall a,b \in \mathbb{K}, \vec{v} \in V : a \cdot (b \cdot \vec{v}) = (ab) \cdot \vec{v} $$
  \item Scalar multiplication satisfies the identity law:
  $$ \forall \vec{v} \in V : 1_{\mathbb{K}} \cdot \vec{v} = \vec{v} \text{ (where $1_{\mathbb{K}}$ is the multiplicative identity in $\mathbb{K}$)} $$
  \item Scalar multiplication distributes over addition:
  $$ \forall a, b \in \mathbb{K}, \vec{v} \in V : (a + b) \cdot \vec{v} = (a \cdot \vec{v}) + (b \cdot \vec{v}) $$
\end{enumerated}

\subsubsection*{Compact definition}

The first four axioms can be replaced by stating that the set $V$ forms an abelian group under the operation of vector addition, with the zero vector ($\vec{0}$) as the identity element.
Axiom 5--8 are identical to the requirement that there exists a ring homomorphism
$$ f : \mathbb{K} \rightarrow \End(V) $$
where $\End(V)$ is the endomorphism ring induced by the the group $V,$ that is:
\begin{enumerated}
  \item The set of endomorphisms; $\varphi : V \rightarrow V$
  \item Addition, defined as pointwise addition of functions; $ \left[\varphi + \psi\right](\vec{v}) = \varphi(\vec{v}) + \psi(\vec{v})$
  \item Multiplication, defined as function composition; $ \varphi\psi = \varphi \circ \psi$
\end{enumerated}
We can then define scalar multiplication in terms of this homomorphism, as
$$ a \cdot \vec{v} = [f(a)](\vec{v}). $$
To simplify notation, we allow ourselves to write, e.g., $f_k$ to mean $f(k)$.
Now, as it turns out, law 5--7 above correspond directly to the ring homomorphism axioms:
$$
f_{a + b} = f_a + f_b \ (5) \quad \quad
f_{ab} = f_a \circ f_b \ (6) \quad \quad
f_1 = \id \ (7)
$$
From this first identity, and by the definition of addition in $\End(V)$, we get
$$ f_{a + b}(\vec{v}) = [f_a + f_b](\vec{v}) = f_a(\vec{v}) + f_b(\vec{v}) $$
which gives us the eighth and last axiom as well.

So, to summarize, a vector space is:
\begin{itemized}
  \item A field $\mathbb{K}$
  \item An abelian group $V$
  \item A homomorphism $f : \mathbb{K} \rightarrow \End(V)$
\end{itemized}

\subsubsection*{Generalization to rings}
The notion of a \begin{em}module\end{em} is a generalization of vector spaces, in which the scalars are elements of any unital ring (i.e., not necessarily a field).

\subsection*{Linear map}
If $V$ and $W$ are vector spaces defined over $\mathbb{K}$, a function $f : V \rightarrow W$ is a \begin{em}linear map\end{em} if it preserves the vector space structure under the two operations; i.e,
\begin{enumerated}
  \item Vector addition: $ \forall \vec{u}, \vec{v} \in V : f(\vec{u} + \vec{v}) = f(\vec{u}) + f(\vec{v}) $; and
  \item Scalar multiplication: $ \forall a \in \mathbb{K}, \vec{v} \in V : f(a \cdot \vec{v}) = a \cdot f(\vec{v}). $
\end{enumerated}
In other words, a linear map is a homomorphism between vector spaces.

\subsection*{Bilinear map}
Given three vector spaces, $U, V, $ and $W,$ all defined over a a field $\mathbb{K}$, a \begin{em}bilinear\end{em} map is a function $b : U \times V \rightarrow W$ which is linear in both arguments. That is, $b$ is linear with respect to;
\begin{itemized}
  \item Addition, in the first argument: $ \forall \vec{u}, \vec{x} \in U, \vec{v} \in V : b(\vec{u} + \vec{x}, \vec{v}) = b(\vec{u}, \vec{v}) + b(\vec{x}, \vec{v}) $
  \item Addition, in the second argument: $ \forall \vec{u} \in U, \vec{v}, \vec{y} \in V : b(\vec{u}, \vec{v} + \vec{y}) = b(\vec{u}, \vec{v}) + b(\vec{u}, \vec{y}) $
  \item Scalar multiplication: $ \forall c \in \mathbb{K}, \vec{u} \in U, \vec{v} \in V : b(c \cdot \vec{u}, \vec{v}) = c \cdot b(\vec{u}, \vec{v}) = b(\vec{u}, c \cdot \vec{v}) $
\end{itemized}

\subsection*{Algebra over a field}
If $\mathbb{K}$ is a field, and $V$ a vector space over $\mathbb{K}$ equipped with a bilinear map $\cdot : V \times V \rightarrow V$, then $V$ is called an \begin{em}algebra over\end{em} $\mathbb{K}$, or $K$-algebra for short.

\subsubsection*{Example}
The complex numbers form an algebra over $\mathbb{R}$, with the bilinear map given as multiplication of two complex numbers, defined as:
$$
(a + b\iu) \cdot (c + d\iu) = (ac - bd) + (ad + bc)\iu
$$

\end{document}
